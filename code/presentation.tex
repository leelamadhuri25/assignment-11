\documentclass[9pt]{beamer}

%\usepackage[utf8x]{inputenc}

\usetheme{Ampang}
\usecolortheme{ampangcolor}
%\usecolortheme{warna}

%%% XeLaTeX engine for Ubuntu Font support
\usepackage{xltxtra}
\usepackage{graphicx}
\setsansfont[
BoldFont=Ubuntu-Bold.ttf,
ItalicFont=Ubuntu-Italic.ttf,
BoldItalicFont=Ubuntu-BoldItalic.ttf
]
{Ubuntu-Regular.ttf}
\setmonofont{UbuntuMono-Regular.ttf}


\title{ASSIGNMENT 10}
\author{LEELA MADHURI}
\date{JAN 5 , 2021}

\begin{document}

\maketitle

\section{QUESTION}

\begin{frame}{QUESTION AND FIGURE}
The input frequency for the given counters 1 MHz,\\ the output frequency observes at Q4 is _______
\includegraphics[width=\textwidth]{picture 1.png}


\end{frame}

\section{Results}

\begin{frame}{TIMING DIAGRAM AND SOLUTION}
\includegraphics[width=1\textwidth]{pic2.png}
The time period doubles for very successive pass from one flip-flop to other.\\
  Let the initial time period and frequency be T,F as the time period is getting doubles so
  time period at $Q_1$ =2T\\\
  Similarly at $Q_2$=4T ; at $Q_3$=8T ; at $Q_4$=16T 
\end{frame}
\begin{frame}{CONTINUATION OF SOLUTION}
so the time period is getting increased in the form of $2^{n}T$ where n can take the value of required output.\\\\
So,frequency at $Q_4$ can be F=$\frac{1}{T \  at \  Q_4 }$\\\
F=$\frac{1}{16}$\\\ {(as initially F=1MHz so T at initial=1 sec)}\\\
Also frequency can be written as F=$\frac{1}{2^4}=\frac{1}{16}$=62.5KHz


  
    
\end{frame}


\end{document}
